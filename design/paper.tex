\documentclass{sigchi}

\CopyrightYear{2019}
\setcopyright{rightsretained}
\doi{http://dx.doi.org/10.475/123_4}
\isbn{123-4567-24-567/08/06}
\conferenceinfo{NA}{TBA}
\acmPrice{\$0.00}

\usepackage{xcolor}
\usepackage{balance}
\usepackage{graphics}
\usepackage[T1]{fontenc}
\usepackage{txfonts}
\usepackage{mathptmx}
\usepackage[pdflang={en-US},pdftex]{hyperref}
\usepackage{color}
\usepackage{booktabs}
\usepackage{textcomp}

\usepackage{microtype}
\usepackage{ccicons}

\def\plaintitle{The Gamma}
\def\plainauthor{First Author, Second Author, Third Author,
  Fourth Author, Fifth Author, Sixth Author}
\def\emptyauthor{}
\def\plainkeywords{Authors' choice; of terms; separated; by
  semicolons; include commas, within terms only; required.}
\def\plaingeneralterms{Documentation, Standardization}

\makeatletter
\def\url@leostyle{%
  \@ifundefined{selectfont}{
    \def\UrlFont{\sf}
  }{
    \def\UrlFont{\small\bf\ttfamily}
  }}
\makeatother
\urlstyle{leo}

\def\pprw{8.5in}
\def\pprh{11in}
\special{papersize=\pprw,\pprh}
\setlength{\paperwidth}{\pprw}
\setlength{\paperheight}{\pprh}
\setlength{\pdfpagewidth}{\pprw}
\setlength{\pdfpageheight}{\pprh}

\definecolor{todoColor}{rgb}{0.6,0.0,0.2}
\definecolor{linkColor}{RGB}{6,125,233}

\newcommand{\todo}[1]{\textcolor{todoColor}{[\textbf{todo}: #1]}}

\hypersetup{%
  pdftitle={\plaintitle},
% Use \plainauthor for final version.
%  pdfauthor={\plainauthor},
  pdfauthor={\emptyauthor},
  pdfkeywords={\plainkeywords},
  pdfdisplaydoctitle=true, % For Accessibility
  bookmarksnumbered,
  pdfstartview={FitH},
  colorlinks,
  citecolor=black,
  filecolor=black,
  linkcolor=black,
  urlcolor=linkColor,
  breaklinks=true,
  hypertexnames=false
}
\begin{document}

\title{\plaintitle}

\numberofauthors{1}
\author{%
  \alignauthor{Anonymous Authors\\
    \affaddr{Unseen University}\\
    \affaddr{Discworld}\\
    \email{e-mail address}}\\
}

\maketitle

\begin{abstract}
TBA
\end{abstract}

% \category{H.5.m.}{Information Interfaces and Presentation
%   (e.g. HCI)}{Miscellaneous} \category{See
%   \url{http://acm.org/about/class/1998/} for the full list of ACM
%   classifiers. This section is required.}{}{}
%
% \keywords{\plainkeywords}

\section{Introduction}

some background

\subsection{Example}

put some screenshot here?

\subsection{Our Approach and Contributions}
We present blah blah.
Our contributions are:

\begin{itemize}
\item Novel language design based on design principles
\item Open source implementation on the web with cool use cases
\item Critical evaluation
\end{itemize}

Long-term goal of making journalism better

\section{Related work}

\textbf{Visual tools.}

\textbf{Programming tools.}
Notebooks

\textbf{Journalism.}
Idyll

\section{Overview}

some walkthrough illustrating thegamma with code and screenshots

\newpage
~
\newpage

\section{Design principles}
TheGamma is based on design principles that we identify in this section. We start by careful
consideration of our target application domain, i.e.~data analyses of open data produced by
journalists and published by online media. By considering the users, challenges and typical
scenarios in this domain, we derive more technical design principles for our system.

\subsection{Principles addressing challenges in journalism}
Modern journalism faces many challenges \todo{cite something}. There are many responses to those
challenges. To develop a new kind of trust, journalists are increasingly presenting not just
outcomes of their analysis, but also the process they used. To develop relationship with readers,
journalists are increasingly looking for meaningful ways of engagement. Those developments
inform the following design principles of TheGamma system. \todo{rewrite this with
more references}

\textbf{A1. Trust through transparency.} The system should allow fact checking of the analyses to
build trust. This means that viewers should be able to determine what is the source of analysed
data and how has the data been transformed.

\textbf{A2. Opening the process.} Journalists are increasingly opening the way they work
to readers in order to build trust. The ability to view how an analysis has been constructed should
not be limited just to experts (say, by running a Jupyter Notebook), but should be
available to all interested readers.

\textbf{A3. Providing meaningful engagement mechanism.} The system should provide a
mechanism through which readers can engage in a meaningful discussion. For example, it should allow
modifying parameters of a data visualization in order to show how, e.g. different choice of
countries affects the result.

\subsection{Principles for open data journalism tool design}
One important observation from the above list is that our tool should be accessible to non-expert
users such as readers and non-technical journalists, while providing extra capabilities for more
technical users. It should be easy to use if one just wants to modify existing code and should
encourage experimentation. Considering these challenges, we identify the following technical
design principles. \todo{There must be papers on learning programming that can be referenced
here.}

\textbf{B1. Learning from examples and by experimentation.} We should support two ways of learning.
Users of tools such as spreadsheets
often learn by looking at existing problem solutions \todo{Advait's PPIG}. Our design should allow
this by making it possible to inspect and retrace steps used while solving a problem in an existing
application. Another principle of spreadsheets that we want to keep is the ability to experiment
and see results immediately. Our design should allow users to try invoking an operation or modifying
a parameter and quickly see if this leads to the desired results.

\textbf{B2. Choice over construction.} To minimize the amount of information that users have to
learn and remember, our system should work in a way that allows constructing programs by
choosing from options that can reasonably appear in a current context, rather than requiring
users to recall particular syntax or exact identifier name.
\todo{recognition over recall?}

\textbf{B3. Make simple things easy and complex things possible.} Some users of the system may, over time, become advanced users
and the system should support those. In other words, the upper bound on what can be achieved
should be well above the most common use cases. At the same time, the complex features that power users might
need should not affect the most elementary uses of the system and should remain completely hidden
until needed. In other words, the lower bound on what one needs to know to use the system for basic
tasks should be as low as possible. \todo{I think I got this idea of "boundaries" on what
is possible from some paper, but cannot recall which...}

\textbf{B4. Visibility of state.} To support transparency, the system should make its entire state
transparent -- when reviewing a data analysis, all parameters should be immediately visible and the
user should not need to, e.g.~navigate through complex user interface to find them.

\section{Systematic description}
Describe in some more systematic detail how things work

\subsection{Language}

\subsection{Type providers}

~

\section{Design principles evaluation}
Evaluate the system with respect to the design principles

~

\section{System evaluation}
More discussion and evaluation, featuring some numbers and case studies

\subsection{Something measurable}

\subsection{Case studies}

\subsection{Scalability}

~

\section{Discussion}

\subsection{Study limitations}
exploratory in nature so we do not make any quantitative claims about effects

not comparing against other systems

\subsection{Design principles}
How well did we do wrt design principles?

\subsection{Design issues}
future challenges and limitations of the model - such as issues when modifying code
in the middle of the call chain

~

~

TODO: Add evaluation according to the evaluating systems paper list

\balance{}
\bibliographystyle{SIGCHI-Reference-Format}
\bibliography{sample}
\end{document}
